\subsection{SPARQL Queries}
Some examples presented in Section 1 describe how an application would access the concepts defined into the movie ontology.
Therefore, SPARQL queries are implemented in order to make a demonstration of how these examples previously proposed work in practice.

Jena API was used to code in Java a routine composed by the following steps:

\begin{itemize}
  \item Load the model from an .owl file into an instance of com.hp.hpl.jena.ontology.OntModel;
\begin{lstlisting}
OntModel model = ModelFactory.createOntologyModel(OntModelSpec.OWL_DL_MEM_RDFS_INF);
// use the class loader to find the input file
InputStream in = FileManager.get().open(inputOWLFileName);
// read the RDF/XML/OWL file
model.read(new InputStreamReader(in), "");
\end{lstlisting}

  \item Write strings that are queries in SPARQL and submit them to com.hp.hpl.jena.query.QueryExecutionFactory;
\begin{lstlisting}
String query2 = "SELECT ?p ?genre WHERE" +
" {?p <http://localhost/WTFDL/movie_ontology.owl#title> \"Lord of the Rings\"." +
"?p <http://www.movieontology.org/2009/10/01/movieontology.owl#belongsToGenre> ?genre.}";
\end{lstlisting}

  \item Collect the results from a com.hp.hpl.jena.query.ResultSet.
\begin{lstlisting}
http://localhost/WTFDL/movie_ontology.owl#Lord_of_the_Rings
http://localhost/WTFDL/movie_ontology.owl#Adventure
\end{lstlisting}

\end{itemize}

Futhermore, some other queries listed below were implemented:

%\begin{itemize}
% \item Get movie starred by Jim Carrey.
%    Query : `` SELECT ?p WHERE ()  ''
%    Result:
%
%\end{itemize}
